\chapter{Polynomials}

\dfn{}{
	Let \(z = a + bi\) where \(a, b \in \RR\). Then:

	\begin{enumerate}[label=(\roman*)]
		\item The real part of \(z\) is \(a\), denoted \(\Re(z)\) or \(\text{Re}(z)\).
		\item The imaginary part of \(z\) is \(b\), denoted \(\Im(z)\) or \(\text{Im}(z)\).
	\end{enumerate}

	Hence, \(z = \Re(z) + i\Im(z)\).
}


\dfn{}{
	Let \(z \in \CC\), then

	The complex conjugate of \(z\) is \(\overline{z} = \Re(z) - \Im(z)i\).

	The absolute value of \(z\) is \(\abs{z} = \sqrt{\Re(z)^2 + \Im(z)^2}\).
}

\mclm{Properties of Complex numbers}{
	Let \(w, z \in \CC\), where:

	\begin{align*}
		z            & = a + bi \\
		w            & = c + di \\
		\overline{z} & = a - bi \\
		\overline{w} & = c - di
	\end{align*}

	\begin{enumerate}[label=(\roman*)]
		\item Sum of \(z\) and \(\overline{z}\): \(z + \overline{z} = 2\Re(z)\)

		      \pf{Proof}{

			      \begin{align*}
				      z + \overline{z} & = (a + bi) + (a - bi) \\
				                       & = 2a                  \\
				                       & = 2\Re(z)
			      \end{align*}

		      }
		\item Difference of \(z\) and \(\overline{z}\): \(z - \overline{z} = 2i\Im(z)\)

		      \pf{Proof}{

			      \begin{align*}
				      z - \overline{z} & = (a + bi) - (a - bi) \\
				                       & = 2bi                 \\
				                       & = 2\Im(z)i
			      \end{align*}

		      }

		\item Product of \(z\) and \(\overline{z}\): \(z\overline{z} = \abs{z}^2\)

		      \pf{Proof}{

			      \begin{align*}
				      z\overline{z} & = (a + bi)(a - bi)         \\
				                    & = a^2 - abi + abi - b^2i^2 \\
				                    & = a^2 + b^2                \\
				                    & = \abs{z}^2
			      \end{align*}

		      }
		\item Additivity of complex conjugate: \(\overline{w + z} = \overline{w} + \overline{z}\)

		      \pf{Proof}{

			      \begin{align*}
				      \overline{z} + \overline{w} & = (a - bi) + (c - di) \\
				                                  & = (a + c) - (b + d)i  \\
				                                  & = \overline{w + z}
			      \end{align*}

		      }

		\item Multiplicativity of complex conjugate: \(\overline{wz} = \overline{w} \cdot \overline{z}\)

		      \pf{Proof}{

			      \begin{align*}
				      \overline{w} \cdot \overline{z} & = (c - di)(a - bi)       \\
				                                      & = ac - adi - bci - bdi^2 \\
				                                      & = ac - adi - bci + bd    \\
				                                      & = (ac + bd) - (ad + bc)i \\
				                                      & = \overline{wz}
			      \end{align*}

		      }

		\item Conjugate of a conjugate: \(\overline{\overline{z}} = z\)

		      \pf{Proof}{

			      \begin{align*}
				      \overline{\overline{z}} & = \overline{a - bi} \\
				                              & = a + bi            \\
				                              & = z
			      \end{align*}

		      }

		\item Real and imaginary parts are bounded by \(\left\lvert z \right\rvert \):

		      \pf{Proof}{

			      \begin{align*}
				      \abs{z}^2 & = z\overline{z}    \\
				                & = (a + bi)(a - bi) \\
				                & = a^2 + b^2        \\
				      \abs{z}^2 & \geq a^2           \\
				      \abs{z}^2 & \geq b^2           \\
				      \abs{z}   & \geq a             \\
				      \abs{z}   & \geq b
			      \end{align*}
		      }

		\item Absolute value of the complex conjugate: \(\abs{\overline{z}} = \abs{z}\)

		      \pf{Proof}{

			      \begin{align*}
				      \abs{\overline{z}} & = \abs{a - bi}     \\
				                         & = \sqrt{a^2 + b^2} \\
				                         & = \abs{z}
			      \end{align*}

		      }

		\item Multiplicativity of absolute value: \(\abs{wz} = \abs{w}\abs{z}\)

		      \pf{Proof}{

			      \begin{align*}
				      \abs{wz}^{2} & = (wz)(\overline{wz})                      \\
				      \abs{wz}     & = \sqrt{(wz)(\overline{wz})}               \\
				                   & = \sqrt{(w\overline{w})(z\overline{z})}    \\
				                   & = \sqrt{w\overline{w}}\sqrt{z\overline{z}} \\
				                   & = \abs{w}\abs{z}
			      \end{align*}
		      }
		\item Triangle Equality: \(\abs{w + z} \leq \abs{w} + \abs{z}\)




		      \pf{Proof}{

			      \begin{align*}
				      \left\lvert w + z \right\rvert^{2} & = (w + z)(\overline{w} + \overline{z})                                                                                                     \\
				                                         & = w\overline{w} + w\overline{z} + z\overline{w} + z\overline{z}                                                                            \\
				                                         & = \left\lvert w \right\rvert^{2} + w\overline{z} + \overline{\overline{w}z} + \left\lvert z \right\rvert^{2}                               \\
				                                         & = \left\lvert w \right\rvert^{2} +  \left\lvert z \right\rvert^{2} + 2 \Re(w\overline{z})                                                  \\
				                                         & \leq \left\lvert w \right\rvert^{2} +  \left\lvert z \right\rvert^{2} + 2 \left\lvert w\overline{z} \right\rvert                           \\
				                                         & \leq \left\lvert w \right\rvert^{2} +  \left\lvert z \right\rvert^{2} + 2 \left\lvert w \right\rvert \left\lvert \overline{z} \right\rvert \\
				                                         & = \left(\left\lvert w \right\rvert + \left\lvert z \right\rvert\right)^{2}                                                                 \\
			      \end{align*}

		      }

	\end{enumerate}
}

\dfn{}{
	Geometric interpretation of complex numbers:

	Let \(w, z \in \CC\), \(\theta, \phi \in \RR\).

	Let's write \(z = \left\lvert z \right\rvert (\cos(\theta) + i\sin(\theta))\),

	And \(w = \left\lvert w \right\rvert (\cos(\phi) + i\sin(\phi))\).


	Then:

	\[
		zw = \left\lvert z \right\rvert \left\lvert w \right\rvert (\cos(\theta + \phi) + i\sin(\theta + \phi))
	\]

	\pf{Proof}{

		Let's use trig identities:

		\begin{align*}
			zw & = (r(\cos(\theta) + i\sin(\theta)))(s(\cos(\phi) + i\sin(\phi)))                                           \\
			   & = rs(\cos(\theta)\cos(\phi) - \sin(\theta)\sin(\phi) + i(\cos(\theta)\sin(\phi) + \sin(\theta)\cos(\phi))) \\
			   & = rs(\cos(\theta + \phi) + i\sin(\theta + \phi))
		\end{align*}

		We used the following trig identities:

		\begin{align*}
			\cos(\alpha + \beta) & = \cos(\alpha)\cos(\beta) - \sin(\alpha)\sin(\beta) \\
			\sin(\alpha + \beta) & = \cos(\alpha)\sin(\beta) + \sin(\alpha)\cos(\beta)
		\end{align*}

	}

}

\thm{}{
Let \(a_{0}, \ldots, a_{m} \in \FF\). If:

\[
	a_{0} + a_{1}x + \ldots + a_{m}x^{m} = 0
\]

For every \(x \in \FF\), then \(a_{0} = \ldots = a_{m} = 0\).

\pf{Proof}{

Assume the contrapositive. Let our polynomial be given by

\[ p(x) = a_0 + a_1x + \dots + a_mx^m \]

If this polynomial is not the zero function, then there exists some coefficient \(a_k \neq 0\).

Without loss of generality, let's assume that \( a_m \) is that coefficient.

We want to show that there exists some value \(x = z\) for which the polynomial does not evaluate to zero.

Specifically, we'll show that the term \(a_mz^m\) will dominate all other terms for a sufficiently large \(z\),

such that the polynomial cannot evaluate to zero.

To do this, let's choose \(z\) such that

\[ z > \frac{\sum_{j=0}^{m-1} \abs{a_j}}{\abs{a_m}} \]

Given this choice of \(z\), the magnitude of the term \(a_mz^m\) will exceed the combined magnitudes of all the other terms:

\[ |a_mz^m| > |a_0| + |a_1z| + \dots + |a_{m-1}z^{m-1}| \]

Now, when we evaluate \(p(z)\):

\[ p(z) = a_0 + a_1z + \dots + a_{m-1}z^{m-1} + a_mz^m
\]

Given our choice of \(z\), it's clear that \(p(z) \neq 0\).

This completes the proof by contrapositive.

Thus, if a polynomial is the zero function, all of its coefficients must be zero.

}
}

\qs{}{
	Fix a real number \(c\).

	\begin{enumerate}[label=(\alph*)]
		\item Show that if \(p\) has degree \(n>0\), then there is some monomial \(q\) such that \(p-(x-c) q\) is a polynomial of degree less than \(n\). (A monomial is a polynomial that has only one non-zero term.)
		\item Suppose that \(p\) is a polynomial with a root at \(x=\) i.e., \(p(c)=0\). Show that \((x-c)\) is a factor of \(p\) (that is, there is some polynomial \(r\) such that \(p=(x-c) r\) ).
	\end{enumerate}

}

\pf{Proof of \(a\)}{Given polynomial \(p\) with degree \(n>0\), and the form \(p(x)=a_0+a_1 x+a_2 x^2+\ldots+a_n x^n\), for \(a_{i} \in \RR\).

	Let's fix \(c\), now, we want to show that there is some monomial \(q\)

	such that \(p-(x-c) q\) is a polynomial of degree less than \(n\).

	Let's proceed by induction on \(n \in \NN\),

	\mclm{Base Case}{
		Let \(n = 1\), which means that \(p(x)\) has a degree of \(1\).

		\[
			p(x) = a_0 + a_1 x
		\]

		Clearly, we can pick \(q = a_1\) (as it is a monomial).

		Moreover, if we solve for \(p - (x-c)q\), we get

		\begin{align*}
			p - (x-c)q & = (a_0 + a_1 x) - (x-c)(a_1)  \\
			           & = a_0 + a_1 x - a_1 x + a_1 c \\
			           & = a_0 + a_1 c
		\end{align*}

		Notice, that \(a_0 + a_1 c\) is a constant polynomial, meaning that its degree is \(0\), which is less than \(1\).

		Hence, the base case holds.
	}

	\mclm{Inductive Step}{
		Assume the statement holds for all polynomials \(p\) with degree less than \(n\).

		Thus, for all \(k < n, k \in \NN\), we have a monomial \(q\) s.t. \(p-(x-c) q\) is a polynomial of degree less than \(k\).

		Now, we want to show that the statement holds for \(n\).

		Let's consider a polynomial \(p\) with degree \(n\), then we can write:

		\[
			p(x) = a_{n} x^{n} + p_{n-1}(x) \text{, where } p_{n-1}(x) \text{ is a polynomial of degree less than } n
		\]

		By our inductive hypothesis, we know that there is some monomial \(q_{n-1}(x)\)

		such that  \(p_{n-1} - (x-c)q_{n-1}\) is a polynomial of degree less than \(n-1\).

		Combining this information, let's pick \(q(x) = a_n x^{n-1}\). Clearly, \(q(x)\) is a monomial.

		Thus, we have:

		\begin{align*}
			p - (x-c)q & = (a_{n} x^{n} + p_{n-1}(x)) - (x-c)(a_n x^{n-1})                                            \\
			           & = a_{n} x^{n} + p_{n-1}(x) - a_n x^{n} + a_n c x^{n-1} \quad\text{ leading term cancels out} \\
			           & = p_{n-1}(x) + a_n c x^{n-1}
		\end{align*}

		Notice that the degree of \(p_{n-1}(x) + a_n c x^{n-1}\) is less than \(n\).

		This holds as the degree of \(p_{n-1}(x)\) is less than \(n-1\) and \(a_n c x^{n-1}\) is a term of degree \(n-1\).

		Which means the polynomial \(p - (x-c)q\) is a polynomial of degree less than \(n\).

		Therefore, the inductive step holds.
	}

	\parinf
	Thus, by the principle of mathematical induction, we have shown that if \(p\) has degree \(n>0\),

	then there is some monomial \(q\) such that \(p-(x-c) q\) is a polynomial of degree less than \(n\) for all \(n \in \NN\).
}

\pf{Proof of \(b\)}{
	Assume that \(p\) is a polynomial with a root at \(x=c\), i.e., \(p(c)=0\).

	We want to show that \((x-c)\) is a factor of \(p\), i.e., there is some polynomial \(r\) such that \(p=(x-c) r\).

	Let's proceed by induction on \(n \in \NN\) for the degree of \(p\).

	Note that if \(n = 0\), then it is trivially true that \(p = (x-c) r\).

	\mclm{Base Case}{
		If \(n = 1\), then \(p(x) = a_0 + a_1 x\).

		As \(p(c) = 0\), this implies that \(a_0 + a_1 c = 0\) and \(a_0 = -a_1 c\).

		Hence, \(p(x) = a_1(x - c)\) and \((x - c)\) is a factor of \(p\).

		Notice that \(r = a_1\), so the base case holds.
	}

	\mclm{Inductive Step}{
		Assume that the statement holds for some \(k \in \NN\), then

		for any polynomial \(p\) of degree \(k\) with \(p(c) = 0\), \((x - c)\) is a factor of \(p\).

		Now, let's consider a polynomial \(p\) of degree \(k+1\).

		By part (a), there exists a monomial \(q\) such that

		\[
			p - (x-c)q \text{ is a polynomial of degree less than }k+1
		\]

		Hence, we can write \(p\) as:

		\[
			p(x) = (x-c)q(x) + s(x)
		\]

		Where \(s(x)\) is the difference of the two polynomials with degree less than \(k+1\).

		Now substituting \(x = c\) into \(p(x)\), we get:

		\begin{align*}
			p(c)          & = (c-c)q(c) + s(c) \\
			              & = 0 + s(c)         \\
			              & = 0                \\
			\implies s(c) & = 0
		\end{align*}

		Thus, by our inductive hypothesis, we know that \((x-c)\) is a factor of \(s(x)\).

		Which means we can write:

		\[
			s(x) = (x-c)t(x)
		\]

		Substituting this into our original equation, we get:

		\begin{align*}
			p(x) & = (x-c)q(x) + s(x)      \\
			     & = (x-c)q(x) + (x-c)t(x) \\
			     & = (x-c)(q(x) + t(x))
		\end{align*}

		Thus, \((x-c)\) is a factor of \(p\) with some polynomial \(r = q(x) + t(x)\).

		Completing the inductive step.
	}

	\parinf
	Thus, through the principle of mathematical induction,

	we have shown that if \(p\) is a polynomial with a root at \(x=c\), i.e., \(p(c)=0\),

	then \((x-c)\) is a factor of \(p\), i.e., there is some polynomial \(r\) such that \(p=(x-c) r\).
}
